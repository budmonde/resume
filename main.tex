\documentclass{resume}

\author{Budmonde Duinkharjav}
\email{budmonde@gmail.com / budmonde@nyu.edu}
\github{budmonde}
\phone{(650) 505-3525}
\website{https://budmonde.com/}

\begin{document}

\maketitle

\begin{area}{Education}
    \item
        \textbf{New York University}, Brooklyn, NY \hfill \emph{Spring 2021 - present}
        \begin{itemize}
            \item
                \emph{PhD Candidate} in Computer Science
                \begin{itemize}
                    \item Advisor: Qi Sun
                \end{itemize}
        \end{itemize}
    \item
        \textbf{Massachusetts Institute of Technology}, Cambridge, MA \hfill \emph{Fall 2014 - Spring 2019}
        \begin{itemize}
            \item
                \emph{MEng} in Computer Science and Engineering
                \begin{itemize}
                    \item Advisor: Fr\'edo Durand
                    \item Thesis: Learning non-stationary SVBRDFs using GANs and Differentiable Rendering
                \end{itemize}
            \item 
                \emph{BS} in Computer Science and Engineering
        \end{itemize}
\end{area}

\begin{area}{Work Experience}
    \item
        \textbf{Adobe Research}, San Jose, CA -
        \emph{Research Intern} \hfill \emph{Summer 2023}
    \item
        \textbf{NVIDIA Research}, Santa Clara, CA - \emph{Research Intern} \hfill \emph{Summer 2022}
        \begin{itemize}
            \item Developed a perceptually-based image quality assessment metric for video game applications.
        \end{itemize}
    \item
        \textbf{Facebook}, Seattle, WA - \emph{Software Engineer} \hfill \emph{Fall 2019 - Spring 2021}
        % Programming Languages \& Runtimes Organization
        \begin{itemize}
            \item Researched and maintained profile-guided optimizations for Facebook's mobile apps.
            \item Contributed to Redex, the java byte-code optimizer for Android apps.
        \end{itemize}
    \item
        \textbf{MIT, CSAIL}, Cambridge, MA - \emph{Research Assistant} \hfill \emph{Fall 2017 - Spring 2019}
        % Computer Graphics Group
        \begin{itemize}
            \item Developed a deep learning system for inferring surface textures using differentiable rendering.
            \item Worked on a system for procedural generation of large-scale city landscape 3D models.
        \end{itemize}
    \item
        \textbf{Facebook}, Menlo Park, CA - \emph{Software Engineering Intern} \hfill \emph{Summer 2018}
        % Software Engineering Intern
        % - Implemented a Spark data pipeline for fetching and aggregating user keyword logs.
    \item
        \textbf{Instagram}, Menlo Park, CA - \emph{Software Engineering Intern} \hfill \emph{Summer 2017}
        % Software Engineering Intern
        % - Worked on Django server request latency optimization and back-end throughput efficiency.
    \item
        \textbf{Omron R\&D}, Kyoto, Japan - \emph{Research Intern} \hfill \emph{Summer 2016}
        % Computer Vision Team
        \begin{itemize}
            \item Worked on super-resolution techniques applied on LIDAR scan images.
        \end{itemize}
    \item
        \textbf{MIT, Civil\&Environ. Eng. Dept.}, Cambridge, MA - \emph{Research Assistant} \hfill \emph{Fall 2014 - Spring 2015}
        % Bourouiba Group
        \begin{itemize}
            \item Analyzed the fluid behavior of water droplet collisions on flat surfaces.
        \end{itemize}
\end{area}

\begin{area}{Publications}
    \item
        \textbf{Color-Perception-Guided Display Power Reduction for Virtual Reality}
        \hfill
        \emph{SIGGRAPH Asia 2022}
        \\%\vspace*{2mm}\\
        \hspace*{2mm}
        \emph{\textbf{B. Duinkharjav*}, K. Chen*, A. Tyagi, J. He, Y. Zhu, Q. Sun (* co-authors)}
        %\emph{ACM Transactions on Graphics (Proceedings of SIGGRAPH Asia 2022, to appear)}
    \item
        \textbf{FoV-NeRF: Foveated Neural Radiance Fields for Virtual Reality}
        \hfill
        \textbf{Best Journal Paper at}
        \emph{ISMAR 2022}
        \\%\vspace*{2mm}\\
        \hspace*{2mm}
        \emph{N. Deng, Z. He, J. Ye, \textbf{B. Duinkharjav}, P. Chakravarthula, X. Yang, Q. Sun}
        %\emph{IEEE Transactions on Visualization and Computer Graphics (Proceedings of ISMAR 2022)}
    \item
        \textbf{Image Features Influence Reaction Time:}
        \hfill
        \textbf{Best Paper at}
        \emph{SIGGRAPH 2022}\\
        \textbf{A Learned Probabilistic Perceptual Model for Saccade Latency}
        \\%\vspace*{2mm}\\
        \hspace*{2mm}
        \emph{\textbf{B. Duinkharjav}, R. Brown, P. Chakravarthula, A. Patney, Q. Sun}
        %\emph{ACM Transactions on Graphics (Proceedings of SIGGRAPH 2022)}
    \item
        \textbf{Modeling And Optimizing Human-In-The-Loop Visual
        Perception}
        \hfill
        \emph{SID Display Week 2022}\\
        \textbf{Using Immersive Displays: A Review}
        \\%\vspace*{2mm}\\
        \hspace*{2mm}
        \emph{Q. Sun, \textbf{B. Duinkharjav}, A. Patney}
        %\emph{SID Display Week 2022}
    \newpage
    \item
        \textbf{Instant Reality: Gaze-Contingent Perceptual Optimization}
        \hfill
        \emph{IEEE VR 2022}\\
        \textbf{for 3D Virtual Reality Streaming}
        \\%\vspace*{2mm}\\
        \hspace*{2mm}
        \emph{S. Chen, \textbf{B. Duinkharjav}, X. Sun, L. Wei, S. Petrangeli, J. Echevarria, C. Silva, Q. Sun}
        %\emph{IEEE Transactions on Visualization and Computer Graphics (Proceedings of IEEE VR 2022)}
    \item
        \textbf{Learning Non-stationary SVBRDFs using GANs and Differentiable Rendering}
        \hfill
        \emph{MIT M.Eng Thesis 2019}
        \\%\vspace*{2mm}\\
        \hspace*{2mm}
        \emph{\textbf{B. Duinkharjav}}
        %\emph{MIT EECS M.Eng Thesis, 2019}
\end{area}

\begin{area}{Teaching Experience}
    \item
        \textbf{Digital and Computational Photography (6.815)}, MIT, Cambridge, MA - \emph{Teaching Assistant} \hfill \emph{Spring 2019}
        \begin{itemize}
            \item Graduate course popular for students focusing in computer graphics, computer vision, and HCI.
            \item \emph{Topics:} Image denoising, demosaicing, stitching, and blending. HDR and panorama photography.
            \item Introduces the HALIDE language for high-performance image processing.
            \item I helped develop some homework assignments, held office hours, and graded assignments.
        \end{itemize}
    \item
        \textbf{Computer Systems Security (6.858)}, MIT, Cambridge, MA - \emph{Teaching Assistant} \hfill \emph{Spring 2018}
        \begin{itemize}
            \item Graduate course popular for students focusing in computer systems.
            \item \emph{Topics:} OS security, capabilities, language security, security in web applications and more.
            \item I held office hours, and graded assignments and final projects.
        \end{itemize}
    \item
        \textbf{WebLab: Intro to Web Programming (6.148)}, MIT, Cambridge, MA - \emph{Co-Instructor} \hfill \emph{Winter 2016, '17, '18}
        \begin{itemize}
            \item Introduces undergraduate students on how to build a dynamic web application with a server backend.
            \item Course culminates in a competition for the best final project. Course website: \texttt{weblab.mit.edu}
            \item I organized the course content and provided technical and creative feedback for student projects.
        \end{itemize}
\end{area}

\begin{area}{Professional Services}
    \item \textbf{Reviewer for} IEEE ISMAR, IEEE VR
\end{area}

%\begin{area}{Projects}
%    \item
%        \textbf{Rendering based optimization method for planar surface embedding} \hfill \emph{Spring 2019} \linebreak
%        \emph{Course Project: Shape Analysis (6.838)}
%        \begin{itemize}
%            \item Implemented a method for embedding a 3D surface triangle-mesh onto a flat surface using a novel method that utilizes differentiable rendering.
%        \end{itemize}
%    \item
%        \textbf{Realistic Rain Simulation} \hfill \emph{Fall 2017} \linebreak
%        \emph{Course Project: Computer Graphics (6.837)}
%        \begin{itemize}
%            \item Implemented a physical simulation algorithm that simulates water droplets flowing down a transparent vertical plane
%            \item Implementation was based on the paper \emph{A heuristic approach to the simulation of water drops and flows on glass panes} by \emph{Chen et al., 2013} 
%        \end{itemize}
%    \item
%        \textbf{RedditBot} \hfill \emph{Fall 2016} \linebreak
%        \emph{Course Project: Machine Learning (6.867)}
%        \begin{itemize}
%            \item Implemented an LSTM network using \emph{Torch} and trained it to emulate comments from the social media platform \emph{Reddit}.
%        \end{itemize}
%\end{area}

\begin{area}{Awards}
    \item \textbf{Snap Research Fellowship, 2022}, Honorable Mention \hfill \emph{Fall 2022}
    \item \textbf{ACM SIGGRAPH 2022}, Best Paper Award \hfill \emph{Summer 2022}
    \item \textbf{MIT Intro to Computer Graphics Final Project}, Best Project Honorable Mention \hfill \emph{Fall 2017}
    \item \textbf{MIT Web Programming Competition}, 1\textsuperscript{st} Place \hfill \emph{Winter 2015}
    \item \textbf{45\textsuperscript{th} International Physics Olympiad}, Silver Medal \hfill \emph{Summer 2014}
    \item \textbf{14\textsuperscript{th} Asian Physics Olympiad}, Bronze Medal \hfill \emph{Spring 2014}
    \item \textbf{44\textsuperscript{th} International Physics Olympiad}, Bronze Medal \hfill \emph{Summer 2013}
\end{area}

\end{document}
